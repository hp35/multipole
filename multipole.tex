%
% File: multipole/multipole.tex [plain TeX code]
% Last change: December 26, 2022
%
% Derivation of the expressions for optically active polarization densities,
% from the very basics of the multipole expansion.
%
% Copyright (C) 2022, Fredrik Jonsson
%
\input epsf
\input eplain
\font\ninerm=cmr9
\font\twentyrm=cmr12 at 20 truept
\font\twelvesc=cmcsc10 at 12 truept
\input amssym % to get the {\Bbb E} font (strikethrough E)
\def\document #1 {\hsize=150mm\hoffset=4.6mm\vsize=230mm\voffset=7mm
  \topskip=0pt\baselineskip=12pt\parskip=0pt\leftskip=0pt\parindent=15pt
  \headline={\ifnum\pageno>1\ifodd\pageno\rightheadline\else\leftheadline\fi
    \else\hfill\fi}
  \def\rightheadline{\tenrm{\it #1}
    \hfil{\it\date}}
  \def\leftheadline{\tenrm{\it\date}
    \hfil{\it #1}}
  \noindent~\vskip-60pt\hskip-40pt{\epsfbox{macros/UU_logo_color.eps}}
  \vskip-42pt\hfill\vbox{\hbox{{\it\author}}
  \hbox{{\it\date}}}\vskip 36pt
  \centerline{\twelvesc #1}
  \vskip 24pt\noindent}
\def\section #1 {\medskip\goodbreak\noindent{\bf #1}
  \par\nobreak\smallskip\noindent}
\def\subsection #1 {\medskip\goodbreak\noindent{\it #1}
  \par\nobreak\smallskip\noindent}
\def\iint{\mathop{\int\kern-8pt\int}}
\def\iiint{\mathop{\int\kern-8pt\int\kern-8pt\int}}
\def\oiint{\mathop{\int\kern-8pt\int\kern-13.2pt{\bigcirc}}}
\def\Re{\mathop{\rm Re}\nolimits} % real part
\def\Im{\mathop{\rm Im}\nolimits} % imaginary part
\def\Tr{\mathop{\rm Tr}\nolimits} % quantum mechanical trace
\def\eqq{\mathop{\vbox{\hbox{\hskip2pt?}\vskip-6pt\hbox{=}}}}

%
% Define in which way we want displayed equation numbers to appear in
% the text. In this case, it is preferred to have the equation numbers
% as one single running index, rather than the form otherwise commonly
% found format used in books, (<chapter number>.<equation number>).
%
\def\eqconstruct#1{#1}

%
% Define in which way we want displayed subequation numbers
% to appear in the text.
%
\newcount\subref
\def\eqsubreftext#1#2{%
  \subref = #2           % The space stops a <number>.
  \advance\subref by 96  % `a' is character code 97.
  #1{\rm\char\subref}%
}

\def\date{December 26, 2022}
\def\author{Fredrik Jonsson}
\document{Multipole expansion and optically active nonlinear media}
\vskip24pt

\section{Outline}
In this analysis, the path from the classical multipole expansion of the
potentials over to the polarizability of the medium and interpretation of the
multipole moments as individual contributing terms are as follows.
\medskip
\item{1.}{Derivation of the classical multipole expansion of the potentials due
    to the fixed and static distribution of charge (say, in the classical
    picture of molecule or antenna. This way, we once and for all define what
    the {\it multipole moments} of the distribution is, say monopole, dipole,
    quadrupole, etc. This is useful once we have the externally applied field
    defined, and also particularly useful when it comes to expressing the
    Hamiltonian for the energy stored by the various moments when placed in an
    externally applied electric field. This expansion means expressing the
    inverse distance $|{\bf x}-{\bf x}'|^{-1}$ as a Taylor series in the
    expressions for the static potentials
    $$
      \Phi({\bf x})={{1}\over{4\pi\varepsilon_0}}
        \iiint_{{\Bbb R}^3}{{\rho({\bf x}')}\over{|{\bf x}-{\bf x}'|}}\,dV',
      \qquad
      {\bf A}({\bf x})={{\mu_0}\over{4\pi}}
        \iiint_{{\Bbb R}^3}{{{\bf J}({\bf x}')}\over{|{\bf x}-{\bf x}'|}}\,dV'.
    $$
    where $\Phi({\bf x})$ is the scalar potential and ${\bf A}({\bf x})$ the
    vector potential. Just to summarize, this is the potentials due to the
    charge and current distributions $\rho({\bf x})$ and ${\bf J}({\bf x})$
    as such (and {\it not} the direct potentials from any externally applied
    field).}
\item{2.}{Expansion of the {\it externally applied} electromagnetic field, in
    terms of a Taylor series around the charge distribution. In this case, the
    starting point is the expression for stored energy,
    $$
      W=\iiint_{{\Bbb R}^3}\rho({\bf x})\Phi({\bf x})\,dV,
    $$
    in which we express the {\it externally applied} potential $\Phi({\bf x})$
    as a Taylor series around a suitable center for the charge distribution
    $\rho({\bf x})$. By interpreting the various-order multipole moments of the
    charge distribution from the first step, we end up with a series of
    {\it interaction} terms between the multipole moments and the externally
    applied field.}
\item{3.}{Going from the classical picture to the quantum-mechanical interaction with a classical (non-quantized) external field. We here move from the classically derived expression for the stored energy over to a picture in which we interpret the various multipole moments of the charge distribution as {\it quantum-mechanical operators}, that is to say taking the step
$$
  \eqalign{
    H&=q\Phi(0)
        -{\bf p}\cdot{\bf E}(0)
        -\sum_i\sum_j Q_{ij}Q_{ij}{{\partial E_j(0)}\over{\partial x_i}}
	-\ldots\cr
  &\hskip10pt\to\quad
    \hat{H}_{\rm I}=\hat{q}\Phi(0)
        -\hat{\bf p}\cdot{\bf E}(0)
	-\sum_i\sum_j \hat{Q}_{ij}\hat{Q}_{ij}{{\partial E_j(0)}\over{\partial x_i}}
	-\ldots\cr
  }
$$}
\medskip

\section{Series expansion of the electrostatic potential from the charge
         distribution}
$$
  \Phi({\bf x})={{1}\over{4\pi\varepsilon_0}}
    \iiint_{{\Bbb R}^3}{{\rho({\bf x}')}\over{|{\bf x}-{\bf x}'|}}\,dV',
$$
$$
  f({\bf x}')\equiv{{1}\over{|{\bf x}-{\bf x}'|}}
    ={{1}\over{[(x-x')^2+(y-y')^2+(z-z')^2]^{1/2}}}
$$
$$
  \eqalign{
  f({\bf x}')&=
    f({\bf x}')\big|_{{\bf x}'={\bf 0}}
    +\sum_k x'_k
      {{\partial f({\bf x}')}\over{\partial x'_k}}\bigg|_{{\bf x}'={\bf 0}}
    +{{1}\over{2}}\sum_j\sum_k x'_j x'_k
      {{\partial^2 f({\bf x}')}
        \over{\partial x'_j\partial x'_k}}\bigg|_{{\bf x}'={\bf 0}}
      \cr&\hskip100pt
    +{{1}\over{6}}\sum_i\sum_j\sum_k x'_i x'_j x'_k
      {{\partial^3 f({\bf x}')}
        \over{\partial x'_i\partial x'_j\partial x'_k}}\bigg|_{{\bf x}'={\bf 0}}
    +O(|{\bf x}'|^4)\cr
  }
$$
$$
  \eqalign{
    f({\bf x}')\big|_{{\bf x}'={\bf 0}}&={{1}\over{|{\bf x}|}}\cr
    {{\partial f({\bf x}')}\over{\partial x'_k}}\bigg|_{{\bf x}'={\bf 0}}
      &={{x_k}\over{|{\bf x}|^3}}\cr
    {{\partial^2 f({\bf x}')}
      \over{\partial x'_j\partial x'_k}}\bigg|_{{\bf x}'={\bf 0}}
        &={{3 x_j x_k - \delta_{jk}|{\bf x}|^2}\over{|{\bf x}|^5}}\cr
    {{\partial^3 f({\bf x}')}
      \over{\partial x'_i\partial x'_j\partial x'_k}}\bigg|_{{\bf x}'={\bf 0}}
        &={{15 x_i x_j x_k - 3\varepsilon_{ijk}\delta_{jk}x_k|{\bf x}|^2}
	   \over{|{\bf x}|^7}}\cr
  }
$$

$$
  \eqalign{
  \Phi({\bf x})&={{1}\over{4\pi\varepsilon_0}}\bigg\{
    {{1}\over{|{\bf x}|}}
    \iiint_{{\Bbb R}^3}\rho({\bf x}')\,dV'
    +\sum_k
      {{x_k}\over{|{\bf x}|^3}}
      \iiint_{{\Bbb R}^3}x'_k\rho({\bf x}')\,dV'
      \cr&\hskip100pt
    +{{1}\over{2}}\sum_j\sum_k
      {{3 x_j x_k - \delta_{jk}|{\bf x}|^2}
         \over{|{\bf x}|^5}}
      \iiint_{{\Bbb R}^3} x'_j x'_k\rho({\bf x}')\,dV'
      \cr&\hskip100pt
    +{{1}\over{6}}\sum_i\sum_j\sum_k
      {{15 x_i x_j x_k - 3\varepsilon_{ijk}\delta_{ij}x_k|{\bf x}|^2}
         \over{|{\bf x}|^7}}
      \iiint_{{\Bbb R}^3} x'_i x'_j x'_k\rho({\bf x}')\,dV'
  \bigg\}\cr
  }
$$

\section{Traceless form of the multipole moments}


\section{Series expansion of the electrostatic potential of the externally
         applied field}
The electrostatically stored energy of a charge density $\rho({\bf x})$ in an
externally applied scalar electric potential $\Phi_{\rm ext}({\bf x})$ is
$$
  W=\iiint_{{\Bbb R}^3}\rho({\bf x})\Phi_{\rm ext}({\bf x})\,dV
$$
The series expansion of the potential of the externally applied electric field
is
$$
  \Phi_{\rm ext}({\bf x})
    =\Phi_{\rm ext}({\bf 0})
      +{{\partial \Phi_{\rm ext}({\bf 0})}\over{\partial x_k}} x_k
      +{{1}\over{2}}{{\partial^2\Phi_{\rm ext}({\bf 0})}
                \over{\partial x_j\partial x_k}} x_j x_k
      +{{1}\over{6}}{{\partial^3\Phi_{\rm ext}({\bf 0})}
                    \over{\partial x_i\partial x_j\partial x_k}} x_i x_j x_k
      +O(|{\bf x}|^4)
$$
where multiple occurrences of indices $i,j,k$ are to be interpreted as sums,
following the Einstein convention of summation, and where we should read the
short-hand notation of the derivaties evaluated at the origo as
$$
  {{\partial \Phi_{\rm ext}({\bf 0})}\over{\partial x_k}}
    \equiv{{\partial \Phi_{\rm ext}({\bf x})}\over{\partial x_k}}
        \bigg|_{({\bf x}={\bf 0})},\qquad
  {{\partial^2\Phi_{\rm ext}({\bf 0})}\over{\partial x_j\partial x_k}}
  \equiv{{\partial^2\Phi_{\rm ext}({\bf x})}\over{\partial x_j\partial x_k}}
        \bigg|_{({\bf x}={\bf 0})},
  \qquad\hbox{etc.}
$$
Inserting the series expansion of the externally applied potential, the
resulting stored energy hence becomes
$$
  W=\Phi_{\rm ext}({\bf 0})q
    +p_k{{\partial \Phi_{\rm ext}({\bf 0})}\over{\partial x_k}}
    +{{1}\over{2}}Q_{jk}{{\partial^2\Phi_{\rm ext}({\bf 0})}
              \over{\partial x_j\partial x_k}}
    +{{1}\over{6}}Q_{ijk}{{\partial^3\Phi_{\rm ext}({\bf 0})}
                  \over{\partial x_i\partial x_j\partial x_k}}
    +O(|{\bf x}|^4)
$$
where the net charge $q$ and the components of the dipole $p_k$, quadrupole
$Q_{jk}$, and octopole $Q_{ijk}$ moments are defined by
$$
  \eqalign{
   q\equiv\iiint_{{\Bbb R}^3}\rho({\bf x})\,dV,\quad
   p_k\equiv\iiint_{{\Bbb R}^3}x_k\rho({\bf x})\,dV,\quad
   Q_{jk}\equiv\iiint_{{\Bbb R}^3}x_j x_k\rho({\bf x})\,dV,\quad
   Q_{ijk}\equiv\iiint_{{\Bbb R}^3}x_i x_j x_k\rho({\bf x})\,dV.
  }
$$
Since the electric field strength is defined by the negative gradient of the
scalar potential,
$$
  E_k = -{{\partial\Phi_{\rm ext}}\over{\partial x_k}},
$$
the stored energy is expressed in terms of the electric field strength as
$$
  W=\Phi_{\rm ext}({\bf 0})q
    -p_k E_k({\bf 0})
    -{{1}\over{2}}Q_{jk}{{\partial E_k({\bf 0})}\over{\partial x_j}}
    -{{1}\over{6}}Q_{ijk}{{\partial^2 E_k({\bf 0})}
                  \over{\partial x_i\partial x_j}}
    +O(|{\bf x}|^4)
$$

\section{General multipole expansion}

$$
  \Phi({\bf x})=
  \iiint_{{\Bbb R}^3}{{\rho({\bf x}')}\over{4\pi\varepsilon_0}}
  \sum^{\infty}_{n=0}\sum^{n}_{k=0}(-2)^k{-1/2\choose n}{n\choose k}
  {{|{\bf x}'|^{2(n-k)}({\bf x}'\cdot{\bf x})^k}\over{|{\bf x}|^{2n+1}}}\,dV'
  \eqdef{eq:genexpansion10}
$$

\noindent{\bf Table 1.} List of terms occurring in the general multipole
    expansion~\eqref{eq:genexpansion10} of the electrostatic scalar potential
    due to a charge distribution $\rho({\bf x})$. Following the multipole
    expansion formalism, the terms are grouped according to their power in
    the distance to the observation point from the origo, of orders
    $O(1/|{\bf x}|^{m+1})$ where $m=2n-k$.
\smallskip\noindent
\vbox{\offinterlineskip
  \def\tableline#1#2#3#4#5#6{%
    height4pt &\omit& &\omit& &\omit& &\omit& &\omit& &\omit&\cr
    &#1& &#2& &#3& &#4& &#5& &#6&\cr
    height4pt &\omit& &\omit& &\omit& &\omit& &\omit& &\omit&\cr}
  \hrule
  \halign{
    &\vrule#&\strut\quad\hfil#\hfil\quad\cr
    \tableline{\vbox{\hbox{Class for term}\vskip4pt\hbox{($2^m$-pole)}}}
       {$m=2n-k$}
       {$n$}{$k$}
       {$\displaystyle{(-2)^k{-1/2\choose n}{n\choose k}}$}
       {$\displaystyle{{{|{\bf x}'|^{2(n-k)}({\bf x}'\cdot{\bf x})^k}
                        \over{|{\bf x}|^{2n+1}}}}$}
    \noalign{\hrule}
    \tableline{Monopole}{0}{0}{0}{1}{$1/|{\bf x}|$}
    \noalign{\hrule}
    \tableline{Dipole}{1}{1}{1}{1}
       {${{{({\bf x}'\cdot{\bf x})}/{|{\bf x}|^3}}}$}
    \noalign{\hrule}
    \tableline{Quadrupole}{2}{1}{0}{$-1/2$}
       {${{{|{\bf x}'|^2}/{|{\bf x}|^3}}}$}
    \tableline{}{}{2}{2}{3/2}
       {${{{({\bf x}'\cdot{\bf x})^2}/{|{\bf x}|^5}}}$}
    \noalign{\hrule}
    \tableline{Octopole}{3}{2}{1}{$-3/2$}
       {${{{|{\bf x}'|^2({\bf x}'\cdot{\bf x})}/{|{\bf x}|^5}}}$}
    \tableline{}{}{3}{3}{$5/2$}
       {${{{({\bf x}'\cdot{\bf x})^3}/{|{\bf x}|^7}}}$}
    \noalign{\hrule}
    \tableline{Hexadecapole}{4}{2}{0}{$3/8$}
       {${{{|{\bf x}'|^4}/{|{\bf x}|^5}}}$}
    \tableline{}{}{3}{2}{$-15/4$}
       {${{{|{\bf x}'|^2({\bf x}'\cdot{\bf x})^2}/{|{\bf x}|^7}}}$}
    \tableline{}{}{4}{4}{$35/8$}
       {${{{({\bf x}'\cdot{\bf x})^4}/{|{\bf x}|^9}}}$}
    \noalign{\hrule}
    \tableline{Dotriacontapole}{5}{3}{1}{$15/8$}
       {${{{|{\bf x}'|^4({\bf x}'\cdot{\bf x})}/{|{\bf x}|^7}}}$}
    \tableline{}{}{4}{3}{$-35/4$}
       {${{{|{\bf x}'|^2({\bf x}'\cdot{\bf x})^3}/{|{\bf x}|^9}}}$}
    \tableline{}{}{5}{5}{$63/8$}
       {${{({\bf x}'\cdot{\bf x})^5}/{|{\bf x}|^{11}}}$}
  }
  \hrule
}

$$
  \eqalign{
    \Phi({\bf x})
    &={{1}\over{4\pi\varepsilon_0}}
      \sum^{\infty}_{n=0}\sum^{n}_{k=0}
      (-2)^k{-1/2\choose n}{n\choose k}\iiint_{{\Bbb R}^3}
      {{|{\bf x}'|^{2(n-k)}({\bf x}'\cdot{\bf x})^k}\over{|{\bf x}|^{2n+1}}}
      \rho({\bf x}')\,dV'\cr
    &={{1}\over{4\pi\varepsilon_0}}\bigg\{
        \underbrace{
          {{1}\over{|{\bf x}|}}\iiint_{{\Bbb R}^3}
          \rho({\bf x}')\,dV'
	}_{\hbox{monopole $\sim O(1/|{\bf x}|)$}}
    \cr&\hskip50pt
       +\underbrace{
          {{1}\over{|{\bf x}|^3}}\iiint_{{\Bbb R}^3}
          ({\bf x}'\cdot{\bf x})
          \rho({\bf x}')\,dV'
	}_{\hbox{dipole moment $\sim O(1/|{\bf x}|^2)$}}
    \cr&\hskip50pt
       +\underbrace{
          {{1}\over{2|{\bf x}|^5}}\iiint_{{\Bbb R}^3}
          \Big[3({\bf x}'\cdot{\bf x})^2-|{\bf x}'|^2|{\bf x}|^2\Big]
          \rho({\bf x}')\,dV'
	}_{\hbox{quadrupole moment $\sim O(1/|{\bf x}|^3)$}}
    \cr&\hskip50pt
       +\underbrace{
          {{1}\over{2|{\bf x}|^7}}\iiint_{{\Bbb R}^3}
          \Big[5({\bf x}'\cdot{\bf x})^2-3|{\bf x}'|^2|{\bf x}|^2\Big]
          ({\bf x}'\cdot{\bf x})
          \rho({\bf x}')\,dV'
	}_{\hbox{octopole moment $\sim O(1/|{\bf x}|^4)$}}
    \cr&\hskip50pt
       +\underbrace{
          {{1}\over{8|{\bf x}|^9}}\iiint_{{\Bbb R}^3}
          \Big[
	    35({\bf x}'\cdot{\bf x})^4
	    -30|{\bf x}'|^2|{\bf x}|^2({\bf x}'\cdot{\bf x})^2
	    +3|{\bf x}'|^4|{\bf x}|^4
	  \Big]
          \rho({\bf x}')\,dV'
	}_{\hbox{hexadecapole moment $\sim O(1/|{\bf x}|^5)$}}
    \cr&\hskip50pt
       +\underbrace{
          {{1}\over{8|{\bf x}|^{11}}}\iiint_{{\Bbb R}^3}
          \Big[
	    63({\bf x}'\cdot{\bf x})^4
	    -70|{\bf x}'|^2|{\bf x}|^2({\bf x}'\cdot{\bf x})^2
	    +15|{\bf x}'|^4|{\bf x}|^4
	  \Big]({\bf x}'\cdot{\bf x})
          \rho({\bf x}')\,dV'
	}_{\hbox{dotriacontapole moment $\sim O(1/|{\bf x}|^6)$}}
    \cr&\hskip50pt
       +O(1/|{\bf x}|^7)
      \bigg\}\cr
    &=\sum^{\infty}_{m=0}\Phi^{(m)}({\bf x}),\cr
  }
$$
where we defined the contributing multipole moments as follows, employing the
Einstein convention of summation over repeated indices,
$$
  \eqalignno{
    \Phi^{(0)}({\bf x})&\equiv{{1}\over{4\pi\varepsilon_0}}
          {{1}\over{|{\bf x}|}}\iiint_{{\Bbb R}^3}
          \rho({\bf x}')\,dV'&\cr
    \Phi^{(1)}({\bf x})&\equiv{{1}\over{4\pi\varepsilon_0}}
          {{x_k}\over{|{\bf x}|^3}}\iiint_{{\Bbb R}^3}
          x'_k \rho({\bf x}')\,dV'&\cr
    \Phi^{(2)}({\bf x})&\equiv{{1}\over{4\pi\varepsilon_0}}
          {{1}\over{2|{\bf x}|^5}}
          \Big[3 x_j x_k \iiint_{{\Bbb R}^3} x'_j x'_k \rho({\bf x}')\,dV'
	  - x_j x_j \iiint_{{\Bbb R}^3} x'_k x'_k\ \rho({\bf x}'),dV' \Big]&\cr
    &={{1}\over{4\pi\varepsilon_0}}
          {{(3 x_j x_k - |{\bf x}|^2\delta_{jk})}\over{2|{\bf x}|^5}}
	  \iiint_{{\Bbb R}^3} x'_j x'_k \rho({\bf x}')\,dV'&\cr
  }
$$
We do here avoid using $m$ as subscript of the multipole terms $\Phi^{(m)}$ of
the potential, in order not to confuse such an index with the elements of a
vector.

\section{Multipole expansion in terms of spherical harmonics -- moving to
         traceless moments}
$$
  \Phi({\bf x})=\sum^{\infty}_{l=0}\sum^{l}_{m=-l}
  {{4\pi}\over{2l+1}}q_{lm}{{Y_{lm}(\vartheta,\varphi)}\over{r^{l+1}}}
$$

\section{Hamiltonian in the interaction picture}
Using the Einstein convention of summation, the Hamiltonian in the interaction
picture with a classical, non-quantized field description, is
$$
  \hat{H}_{\rm I}(t) = -\hat{p}_{\alpha}E_{\alpha}(t)
    -{{1}\over{2}}\hat{Q}_{\alpha\beta}
       {{\partial E_{\alpha}(t)}\over{\partial x_{\beta}}}
    -{{1}\over{6}}\hat{Q}_{\alpha\beta\gamma}
       {{\partial^2 E_{\alpha}(t)}\over{\partial x_{\beta}\partial x_{\gamma}}}
    -\hat{m}_{\alpha}B_{\alpha}(t)
$$


\section{Expressing solutions to Poissons equation in terms of spherical
         harmonics}
The spherical harmonic functions are well-known for providing solutions to the
variable-separated form of Laplace equation, $\nabla^2\Phi=0$. However, they
are also useful as a tool when expanding the electrostatic potential from a
charge distribution $\rho({\bf x})$, for which
$\nabla^2\Phi=\rho/\varepsilon_0$.

\subsection{Solving Laplace's equation in spherical coordinates}
Starting with the Laplace operator expressed in spherical
coordinates\numberedfootnote{We here follow the ordering convention
    $(r,\vartheta,\varphi)$ for the radial distance $r$, polar angle
    $\vartheta\in[0,\pi]$ and azimuthal angle $\varphi\in[0,2\pi]$,
    as recommended in the ISO 80000-2:2019 standard {\it Quantities
    and units -- Part 2: Mathematics}, pp. 20--21. Item no. 2-17.3.}
$(r,\vartheta,\varphi)$, for $0\le\vartheta\le\pi$ and $0\le\varphi\le2\pi$,
$$
  \nabla^2\Phi = {{1}\over{r^2}}{{\partial}\over{\partial r}}
    \Big(r^2{{\partial\Phi}\over{\partial r}}\bigg)
    +{{1}\over{r^2\sin^2\vartheta}}{{\partial^2\Phi}\over{\partial\varphi^2}}
    +{{1}\over{r^2\sin\vartheta}}{{\partial}\over{\partial\vartheta}}
      \Big(\sin\vartheta{{\partial\Phi}\over{\partial\vartheta}}\bigg),
   \eqdef{eq:laplace10}
$$
we may for the Laplace equation $\nabla^2\Phi=0$ separate this by
$$
  \Phi(r,\varphi,\vartheta) = R(r)F(\varphi)T(\vartheta)
  \eqdef{eq:laplace20}
$$
into\numberedfootnote{As a reference, see for example, Morse \& Feshbach,
    {\it Methods of Theoretical Physics} (McGraw--Hill, New York, 1953),
    Chapter 10, {\it Solutions of Laplace's and Poisson's Equations}, p.~1264.}
$$
  \eqalignno{
    {{1}\over{R(r)}}{{\partial}\over{\partial r}}
      \Big(r^2{{\partial R(r)}\over{\partial r}}\bigg)&=\lambda=\hbox{const.},
      \eqdefn{eq:laplace30}&\eqsubdef{eq:laplace30a}\cr
    {{1}\over{F(\varphi)}}{{\partial^2 F(\varphi)}\over{\partial\varphi^2}}
    &=-m^2=\hbox{const.}&\eqsubdef{eq:laplace30b}\cr
    {{\sin\vartheta}\over{T(\vartheta)}}
      {{\partial}\over{\partial\vartheta}}
      \Big(\sin\vartheta{{\partial T(\vartheta)}\over{\partial\vartheta}}\bigg)
        +\lambda\sin^2\vartheta&=m^2.
    &\eqsubdef{eq:laplace30c}\cr
  }
$$
We here chose the sign of $-m^2$ and its ``pre-squared'' form already from the
knowledge that $F(\varphi)$ must be a periodic (non-exponential) function, and
that we know that we at some point will use the square root of the constant
when solving the second-order linear ODE. To start with,
Eq.~\eqref{eq:laplace30b} is easily integrated to yield
$$
  F_m(\varphi)=C_m\exp(im\varphi)+D_m\exp(-im\varphi),
$$
where $C_m$ and $D_m$ are constants, and where we for the sake of clarity in
notation added an explicit index on $F_m(\varphi)$. Since $F_m(\varphi)$ must
be a periodic function with a maximum period of $2\pi$, this implies that $m$
must be an integer. Due to the presence of positive as well as negative
exponent in the general solution for $F_m(\varphi)$, we may without lack of
generality remove this redundancy to have the solution for the azimuthal
dependence expressed as
$$
  F_m(\varphi)=C_m\exp(im\varphi),\qquad m=\ldots,-2,-1,0,1,2,\ldots
$$
Evaluation of $F_m(\varphi)$ over $0\le\varphi\le2\pi$ then immediately yields
that these solutions indeed are orthogonal,
$$
  \int^{2\pi}_0 F_m(\varphi)F_{m'}(\varphi)\,\varphi
    = C_m C_{m'}2\pi\delta_{mm'},
$$
though by further applying the requirement that they in addition should be
orthonormal, in which case we relieve the constants $C_m$ from any boundary
conditions, instead lifting this over to the integration constants of other
factors in the variable separation of Eq.~\eqref{eq:laplace20}, we require
that $C_m=(2\pi)^{-1/2}$, and that hence
$$
  F_m(\varphi)={{1}\over{\sqrt{2\pi}}}\exp(im\varphi),
  \qquad m=\ldots,-2,-1,0,1,2,\ldots
  \eqdef{eq:laplace40}
$$
The requirement on $m$ being an integer then leads to Eq.~\eqref{eq:laplace30c}
taking the Sturm-Liouville form
$$
  {{1}\over{\sin\vartheta}}
    {{\partial}\over{\partial\vartheta}}
    \Big(\sin\vartheta{{\partial T(\vartheta)}\over{\partial\vartheta}}\bigg)
    +\Big(\lambda - {{m^2}\over{\sin^2\vartheta}}\Big)T(\vartheta)=0.
  \eqdef{eq:laplace50}
$$
Due to the presence of $\sin\vartheta$ in the denominators, this ODE is
singular at $\vartheta=0$ and $\vartheta=\pi$, something we in just a moment
will address when it comes to constants of integration. Changing variable to
$\xi=\cos\vartheta$, with $0\le\xi\le1$, we then have
$$
  {{\partial}\over{\partial\xi}}
    =-{{1}\over{\sin\vartheta}}{{\partial}\over{\partial\vartheta}},
$$
and Eq.~\eqref{eq:laplace50} hence takes the form of {\it Legendre's associated}
ordinary differential equation as
$$
  (1-\xi^2){{\partial^2 T(\xi)}\over{\partial\xi^2}}
  -2\xi{{\partial T(\xi)}\over{\partial\xi}}
    +\Big(\lambda - {{m^2}\over{1-\xi^2}}\Big)T(\xi)=0.
  \eqdef{eq:laplace60}
$$
Whenever $\lambda=n(n+1)$ with $n\ge m$ is an integer, Legendre's associated
equation has the general solution $T(\xi)=aP^m_n(\xi)+bQ^m_n(\xi)$, where $a$
and $b$ are constants and where $P^m_n(\xi)$ and $Q^m_n(\xi)$ are the respective
first and second kind associated Legendre functions of degree $n$ and order $m$.

Due to the fact that $Q^m_n(\xi)$ is unbounded at $\xi=1$ and $\xi=-1$ (that is
to say, at polar angles $\vartheta=0$ and $\vartheta=\pi$, respectively), the
requirement of a bounded solution hence yields the requirement that $b=0$. The
bounded {\it orthonormal} eigenfunctions\numberedfootnote{For the computation
    of the normalization factor, see for example D.~D.~Trim {\it Applied
    Partial Differential Equations} (PWS-Kent, Boston, 1990), Section 8.6,
    {\it Sturm-Liouville Systems and Legendre's Differential Equation.}}
of the Sturm-Loiuville system of Eq.~\eqref{eq:laplace50} are hence, after
some algebra and by explicitly indexing the eigenfunctions,
$$
  T_{nm}(\vartheta)=
    \sqrt{{{(2n+1)(n-m)!}\over{2(n+m)!}}}P^m_n(\cos\vartheta).
  \eqdef{eq:laplace70}
$$
With this form of coefficient, it is straightforward albeit somewhat cumbersome
to verify that the orthonormality of $T_{mn}(\vartheta)$ yields
$$
  \int^{\pi}_0
    T_{nm}(\vartheta) T_{n'm'}(\vartheta)\sin\vartheta
  \,d\vartheta
  =\delta_{nn'}\delta_{mm'}
$$
Again, by imposing orthonormality on $T_{nm}(\vartheta)$, we effectively
relieve the coefficients from being used to match requirements imposed by
boundary conditions, and we have hence left this to the integration constants
which finally will appear for the radial factor $R(r)$.

Finally, using that $\lambda=n(n+1)$ from the conclusion of the Sturm-Liouville
form of the equation for $T(\vartheta)$, Eq.~\eqref{eq:laplace30a} becomes
$$
  {{\partial}\over{\partial r}}
    \Big(r^2{{\partial R(r)}\over{\partial r}}\bigg)-n(n+1)R(r)=0.
  \eqdef{eq:laplace80}
$$
From the shape of this equation, in particular with its terms of the form
$$
  r^2{{\partial^2 R}\over{\partial r^2}},\quad
  r{{\partial R}\over{\partial r}},\quad\hbox{and}\quad R,
$$
we may immediately conclude that the general solution must be a polynomial in
$r$, and we can directly integrate the radial equation to yield
$$
  R_n(r)=A_n r^n + B_n r^{-(n+1)},
  \eqdef{eq:laplace90}
$$
where $A_n$ and $B_n$ are constants. As the solution for $R_n(r)$ here stands,
we should keep in mind that the constants $A_n$ and $B_n$ in fact implicitly
also depend on the index $m$, as we have normalized the factors $F_m(\varphi)$
and $T_{nm}(\vartheta)$. Thus, by combining the solutions for the factors
$F_m(\varphi)$, $T_{mn}(\vartheta)$ and $R_n(r)$, of Eqs.~\eqref{eq:laplace40},
\eqref{eq:laplace70} and \eqref{eq:laplace90}, respectively, into the form
$\Phi(r,\varphi,\vartheta) = R(r)F(\varphi)T(\vartheta)$ used for the variable
separation, we arrive at the solution\numberedfootnote{Notice that
    J.~D.~Jackson, {\it Classical Electrodynamics} (Wiley, New York, 1975),
    2nd Edn., Eqs.~(3.2) and (3.8), in the treatise on Laplace's equation
    in spherical coordinates already in the separation of variables assume
    an inverse radial dependence, as
    $$
      \Phi(r,\varphi,\vartheta) = {{R(r)}\over{r}}F(\varphi)T(\vartheta),
    $$
    hence providing a slightly more elegant solution as
    $$
      \Phi(r,\vartheta,\varphi) = \sum^{\infty}_{n=0}\sum^{n}_{m=0}
        (A_{nm} r^{n+1} + B_{nm} r^{-n)})Y_{nm}(\vartheta,\varphi).
    $$}
$$
  \Phi(r,\vartheta,\varphi) = \sum^{\infty}_{n=0}\sum^{n}_{m=-n}
  (A_{nm} r^n + B_{nm} r^{-(n+1)})Y_{nm}(\vartheta,\varphi)
  \eqdef{eq:laplace100}
$$
where\numberedfootnote{We may here recall that the ``$4\pi$'' in the
    denominator of the square root stems from the $(2\pi)^{-1/2}$ occurring
    in the azimuthal normalization of $F_m(\varphi)$, multiplied by the ``$2$''
    in the denominator of the normalization constant of $T_{mn}(\vartheta)$.}
$$
  Y_{nm}(\vartheta,\varphi)\equiv
  \sqrt{{{(2n+1)(n-m)!}\over{4\pi(n+m)!}}}\exp(im\varphi)P^m_n(\cos\vartheta)
  \eqdef{eq:laplace110}
$$
are the orthonormal spherical harmonic functions,\numberedfootnote{Morse \&
    Feshbach, {\it Methods of Theoretical Physics} (McGraw--Hill, New York,
    1953), Chapter 10, {\it Solutions of Laplace's and Poisson's Equations},
    p.~1264. See also Jackson ({\it ibid.}), Eq.~(3.53).}
here listed with the arguments in order $(\vartheta,\varphi)$ to conform with
the commonly used convention of notation. The orthonormal condition on
$Y_{nm}(\vartheta,\varphi)$ over the entire enclosing sphere is from the
previously stated orthonormality conditions on $T_{nm}(\vartheta)$ and
$F_m(\varphi)$ easily verified to yield
$$
  \eqalign{
  \int^{\pi}_0
    \int^{2\pi}_0
      &Y_{nm}(\vartheta,\varphi) Y^*_{n'm'}(\vartheta,\varphi)\sin\vartheta
    \,d\varphi
  \,d\vartheta
  \cr
  &=\int^{\pi}_0
      \int^{2\pi}_0
        T_{nm}(\vartheta)F_m(\varphi)
	\bigg(
          T_{n'm'}(\vartheta)F_{m'}(\varphi)
	\bigg)^*
	\sin\vartheta
      \,d\varphi
    \,d\vartheta\cr
  &=\underbrace{
      \int^{2\pi}_0 F_m(\varphi) F^*_{m'}(\varphi)\,d\varphi
      }_{=\delta_{mm'}}
    \underbrace{
      \int^{\pi}_0
        T_{nm}(\vartheta) T^*_{n'm'}(\vartheta) \sin\vartheta
      \,d\vartheta
    }_{=\delta_{nn'}\delta_{mm'}}\cr
  &=\delta_{nn'}\delta_{mm'}\cr
  }
$$

\bye
